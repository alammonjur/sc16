\section{Introduction}
A priority queue is a popular data structure used for various applications such as routing, anomaly prioritization, shortest path search, and scheduling~\cite{ah1,ah2,ah3}.
It is a data structure in which (1) each element has a priority, and (2) a dequeue operation removes and returns the highest priority element from the queue.
%The concept is a basic component for scheduling used in most routers and event-driven simulators \cite{hw1,fpga1}.

A parallel heap is expected to offer higher throughput and quality of service than its serial counterpart.
There are several hardware-based parallel implementations of a priority queue~\cite{hw1,hw2,hw8,hw9,fpga1,fpga2,fpga3} for handling a large volume of elements, though they have individual drawbacks.
The {\it Systolic Arrays} and {\it Shift Registers} based approaches \cite{hw8,hw9}, for example, are not scalable and require $O(n)$ comparators for $n$ nodes.
FPGA-based pipelined heap, presented by Ioannou {\it et. al} \cite{fpga1}, is scalable and can run for 64K nodes, but it stalls parallel {\it insert-delete} operations. Calendar queues \cite{hw1} incur significant hardware cost when supporting a large priority set. A hybrid priority queue architecture proposed by \cite{hwsw1} operates on $O(log \ n)$ time.

All of the existing works, however, share a common drawback: they do not address {\it holes} created as a result of {\it insert} operations in parallel of {\it min-delete} operations. 
Holes occupy storage elements as nodes but do not have valid data, so retaining holes introduces additional overhead.
They also lead to an unbalanced tree, which may result in longer response time.

Toward this end, this paper proposes an efficient hardware implementation of a parallel priority queue.
The contributions of this paper are summarized as follows:
\begin{itemize}
\item {\bf Hole minimization:} The proposed approach minimizes holes created by parallel operations on a priority queue. Our hole minimization technique reduces hardware cost by 37.76\% in terms of number of lookup tables (LUTs) and average response time by 4.48\%.
\item {\bf Hardware sharing:} Sharing hardware between two consecutive pipelined levels reduces the hardware cost. The hardware sharing technique contributes a 3.70\% cost reduction.
\item {\bf Replacement operation:} Upon detection of a min-delete operation immediately followed by an insert operation, a {\it replacement} operation substitutes these two operations. This method reduces average response time by 30.36\%.
\end{itemize}

The rest of this paper is organized as follows:
\autoref{s:background} contains an overview of literature related to this work.
\autoref{s:impl} presents our proposed design with hole minimization. We also propose several optimization techniques for performance efficiency and hardware minimization.
\autoref{s:eval} describes the implementation results along with performance comparisons with existing designs.
\autoref{s:cons} concludes the paper and identifies some directions for future research.

